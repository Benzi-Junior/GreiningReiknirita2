%~ xelatex -shell-escape grrVerkefni2.tex

\documentclass[a4paper,oneside]{article}
\usepackage{a4wide}
\usepackage[icelandic]{babel}
\usepackage{fontspec}
\usepackage{xunicode}
\usepackage{graphicx}
\usepackage{enumerate}

\usepackage{mathtools} % := from \coloneqq
\usepackage{units} % falleg 1/3 brot
\usepackage{amsmath} % \begin{align} (also in \{mathtools})
\usepackage{amsthm} % proof
\usepackage{amssymb} % blacksquare og mathbb
 \renewcommand{\qedsymbol}{$\blacksquare$}

%~ Teikna tré
\usepackage{qtree}

\usepackage{fancyhdr}
\pagestyle{fancy}

\lhead{Greining reiknirita}
\chead{Verkefni 1}
\rhead{Háskóli Íslands}

\title{
    Greining reiknirita:\ Verkefni 1
    \\\small{Kennari: Páll Melsted}
}
\author{Bjarki Geir Benediktsson, \and  Bjarni Jens Kristinsson \and og Tandri Gauksson}
\date{\small{Skil:\ 9. febrúar 2014}}


\begin{document}
\maketitle

\section{Lýsing á gagnagrind}

%%%%%TANDRI
Verkefnið var að hanna gagnagrind sem byggir á einhverskonar tréi og geymir lokuð bil á rauntalnalínunni þar sem báðir endanpunktarnir eru heiltölur.
Notast var við AVL tré, en AVL tré eru tvíleitartré sem uppfylla að auki skilyrðið: \textit{Fyrir sérhvern hnút er mismunur á hæðum hægra og vinstra hluttrés í mesta lagi einn}.
Því skilyrði má viðhalda með því að framkvæma snúninga á tréinu í hvert sinn sem innsetning eða eyðing valda ójafnvægi.\\

Fyrir hnút $N$ látum við $T_N$ tákna tréið sem hefur $N$ sem rót. Í $N$ eru eftirfarandi upplýsingar geymdar:
\begin{itemize}
	\item Lokað bil, $N.I$,
	\item $N.Min =$ Lægsti endapunktur meðal þeirra bila sem eru innihaldin í $T_N$,
	\item $N.Max =$ Hæsti endapunktur meðal þeirra bila sem eru innihaldin í $T_N$,
	\item $N.h = $ Hæð $T_N$,
	\item $N.l = $ Bendir á vinstra barn (ef það er til),
	\item $N.r = $ Bendir á hægra barn (ef það er til).
\end{itemize}

Við segjum að tvö bil skerist ef sniðmengi þeirra er ekki tómt.
Við segjum að bil $J$ skeri tréið $T_N$ ef bilin $J$ og $[N.Min, N.Max]$ skerast (ekkert bil sker tómt tré).
Til þess að finna öll bil í $T_N$ sem skera bilið $J$ var búið til nýtt tré $S$ og eftirfarandi endurkvæma reiknirit notað:
\begin{enumerate}
	\item Ef $N.I$ og $J$ skerast er bilinu $N.I$ fært inn í $S$.
	\item Ef $J$ sker $T_{N.l}$ er reikniritið notað á $T_{N.l}$.
	\item Ef $J$ sker $T_{N.r}$ er reikniritið notað á $T_{N.r}$.
\end{enumerate}
Þegar keyrslu lýkur inniheldur $S$ nákvæmlega þau bil í $T_N$ sem skera $J$.\\

Þetta sama reiknirit má svo nota til þess að útfæra aðrar aðferðir.
Til þess að finna öll bil í $T_N$ sem innihalda tölu $k$ má í staðinn leita að öllum bilum í $T_N$ sem skera bilið $[k,k]$.
Til þess að finna öll bil í $T_N$ sem skera bilið $[a,b]$ má finna tréið $S$ af öllum bilum í $T_N$ sem innihalda töluna $a$ og finna svo öll bil í $S$ sem innihalda töluna $b$.\\


%%%%%%%%BJARKI
Verkefnið var að hanna gagnagrind sem notaðist við einhverskonar tré til að geyma lokuð bil  á rauntalnalínunni þar sem báðir endanpunktarnir eru heiltölur.
Við notuðum AVL tré til að geyma bilin sérhver hnútur inniheldur bil ásamt hæsta og lægsta endapunkt sem fyrirfynnst í undirtrénu með þann hnút sem rót. 
Ef bil I  sker bilið frá lægsta til hæsta endapunktar undirtrés segjum við að bilið skeri undirtréð.

Innsetning, leit og eyðing eru framkvæmd eins og við er að búast af AVL tré.

Þegar leitað er að þeim bilum í trénu sem skera eitthvað bil I þá er athugað hvort I skeri bilið í rótinni 
svo er athugað hvort I skeri undirtrén og endurkvæmt leitað í þeim undirtrjám sem I sker.
Þau bil sem skera I eru sett í nýtt tré af sömu tegund og því tré er svo skilað í lokin.

Klasin sem við notuðum fyrir bil leifði úrkynjuð bil það er $[a,b]$ þar sem $b\leq a$.

þegar leitað er að bilum í trénu sem innihalda ákveðin punkt p  þá er leitað að þeim bilum sem skera  $[p,p]$ 

Þegar leitað er að þeim bilum í trénu sem innihalda bil $[a,b]$ þá er fyrst leitað að þeim bilum sem skera $[a,a]$ og svo er leitað að þeim bilum sem skera  $[b,b]$ í trénu sem skilað var úr fyrri leitinni.

\end{document}
