%~ xelatex -shell-escape grrVerkefni1.tex

\documentclass[a4paper,oneside]{article}
\usepackage{a4wide}
\usepackage[icelandic]{babel}
\usepackage{fontspec}
\usepackage{xunicode}
\usepackage{graphicx}
\usepackage{enumerate}

\usepackage{mathtools} % := from \coloneqq
\usepackage{units} % falleg 1/3 brot
\usepackage{amsmath} % \begin{align} (also in \{mathtools})
\usepackage{amsthm} % proof
\usepackage{amssymb} % blacksquare og mathbb
 \renewcommand{\qedsymbol}{$\blacksquare$}

%~ Teikna tré
\usepackage{qtree}

\usepackage{fancyhdr}
\pagestyle{fancy}

\lhead{Greining reiknirita}
\chead{Verkefni 1}
\rhead{Háskóli Íslands}

\title{
    Greining reiknirita:\ Verkefni 1
    \\\small{Kennari: Páll Melsted}
}
\author{Bjarki Geir Benediktsson, Bjarni Jens Kristinsson og Tandri Gauksson}
\date{\small{Skil:\ 3. febrúar 2014}}


\begin{document}
\maketitle

\section{Verklýsing}
Gefinn er þýddur Java klasi \texttt{Sorter.class} sem hefur að geyma tíu röðunaraðferðir \texttt{s1,...,s10} sem taka allar inn viðfang af taginu \texttt{int[] a} og skila fylkinu aftur í vaxandi röð. Vitað er að aðferðirnar tíu eru eftirfarandi:
\begin{itemize}
    \item Insertion sort
    \item Merge sort
    \item Heap sort
    \item Quick sort með vendistak valið sem
    \begin{itemize}
        \item aftasta stak, n
        \item miðjustak, n/2
        \item miðgildið af fremsta, miðju og aftasta staki
        \item slembistak
    \end{itemize}
    \item Counting sort
    \item Radix sort
    \item Bucket sort
\end{itemize}

Markmið verkefnisins er að finna út hvaða aðferðir úr \texttt{Sorter.class} er hvaða röðunarreiknirit.

\section{Framkvæmd}
Við vitum að Quick sort, Insertion sort og Heap sort aðferðirnar nota ekki auka minni sem fall af inntaksstærð, heldur fast auka minni óháð stærð fylkisins. Hinar aðferðirnar; Merge sort, Counting sort, Radix sort og Bucket sort nota hins vegar auka minni sem er háð stærð fylkisins. \\

\noindent Það er því ljóst að við eigum að fá út að fjórar aferðir nota minni í kös sem fall af fylkjastærð en sex sem nota fast minni, óháð fylkjastærð. \\

\noindent Við bjuggum til 1000 staka fylki\footnote{\texttt{FyrstaGreining.java}} með slembnum gildum og mældum minnisnotkunina (í bætum) í kösinni. Mælingarnar voru þessar: \\ 

\begin{tabular}{ c | r }
    fall         & Minnisnotkun \\ \hline
    \texttt{s1}  &       620352 \\
    \texttt{s2}  &            0 \\
    \texttt{s3}  &            0 \\
    \texttt{s4}  &            0 \\
    \texttt{s5}  &      5447920 \\
    \texttt{s6}  &     43483328 \\
    \texttt{s7}  &            0 \\
    \texttt{s8}  &     28563496 \\
    \texttt{s9}  &      5502592 \\
    \texttt{s10} &            0 \\
\end{tabular} \\

\noindent Til þess að finna út hvaða fjögur föll nota minni háð inntaksstærð skrifuðum við forrit í Java\footnote{\texttt{NotQuick.java}} sem mælir minnisnotkun sem fall af fylkjastærð $n$ og létum $n$ byrja í $10000$ og stökkva um $10000$ þar til minnið kláraðist. Við plottuðum niðurstöðurnar\footnote{\texttt{plot.r}} með \texttt{R} og útkoman varð þessi: \\

\centerline{\includegraphics[scale=0.65]{Minnisnotkun.png}}

Hér sést að \texttt{s1} notar fast minni. Niðurstöður þessa liðs eru því eftirfarandi: \\

\begin{tabular}{ c | c }
    Notar auka minni sem fall af inntaksstærð & Notar fast mikið auka minni \\ \hline
    \texttt{s5} &  \texttt{s1} \\
    \texttt{s6} &  \texttt{s2} \\
    \texttt{s8} &  \texttt{s3} \\
    \texttt{s9} &  \texttt{s4} \\
                &  \texttt{s7} \\
                & \texttt{s10} \\ 
\end{tabular}

%~ 1.2
\subsection{Aðferðir sem nota ekki auka minni}
\noindent Af þessum sex aðferðum sem nota ekki minni á kös sem fall af inntaksstærð þá notar Insertion sort og Heapsort fast minni á hlaða vegna þess að þau eru ekki endurkvæm. \\

\noindent Það kom í ljós að sum fallana kláruðu minnið á hlaðanum\footnote{\texttt{Overflow.java}} fyrir frekar lítil slembifylki (undir $10^5$ stök) en það voru föllin \texttt{s3}, \texttt{s4}, \texttt{s7} og \texttt{s10}. Við drögum því þá ályktun að þetta eru Quick sort aðferðirnar meðan \texttt{s1} og \texttt{s2} eru Insertion sort og Heap sort. \\

\subsubsection{Insertion sort og Heap sort}
%~ Vitað er að \texttt{s1} og \texttt{s2} eru Insertion sort og Heap sort. 
Með því að mæla tímann\footnote{\texttt{Comparer.c12()}} sem það tekur að raða sama $n$ staka fylki með báðum aðferðum, þar sem $n = 2^0 \cdot 1000, ..., 2^{11} \cdot 1000$, og teikna línurit\footnote{\texttt{plot.r}} fyrir niðurstöðurnar má auðveldlega sjá að \texttt{s1} tekur $O(n \log n)$ tíma og að \texttt{s2} tekur $O(n^2)$ .\\

\begin{figure}[h!]
    \centering
    \includegraphics[scale=0.65]{Insertion_og_Heap.png}
    \caption{Hallatala \texttt{s1} = 1.108 \& hallatala \texttt{s2} = 1.992}
\end{figure}

\noindent Höfum því að \texttt{s1} er Heap sort og \texttt{s2} er Insertion sort.
 
\subsubsection{Quick sort með aftasta stak sem vendistak}
Til að greina á milli mismunandi leiða til að velja vendistak í Quick sort byrjuðum við á að skoða hvert þeirra kláraði minnið á hlaðanum fyrir minnsta raðaða fylkið\footnote{\texttt{OrderedTime.java}}. \\

\noindent Fyrir raðað fylki þarf ekki að framkvæma neinar skiptingar en það er versta tilfelli aðferðarinnar sem velur alltaf aftasta stak sem skiptistak en þá mun hlaðinn stækka í línulegu hlutfalli við lengd fylkisins. \\

\noindent Með því að auka alltaf stærð fylkisins var það \texttt{s3} sem fyrst kláraði hlaðann meðan hinar aðferðirnar (verandi allar nokkuð skilvirkar fyrir raðað fylki) kláruðu ekki hlaðann við mælingarnar. Þess vegna er \texttt{s3} Quick sort aðferðin sem notar aftasta stak sem vendistak.

\subsubsection{Quick sort með slembið vendistak}
Því næst athuguðum við \emph{sagtannarfylki}, þ.e.\@ fylki sem er samansett úr röðuðum hlutfylkjum. \\

\noindent Ef fjöldi hlutfylkjanna er slétt tala þá munu miðgildisaðferðin og miðjustaksaðferðin hafa stærsta eða minnsta gildi sem vendistak og eru því mjög óhentugar fyrir það tilfelli. Aftur á móti ef allt inntaksfylkið er raðað þá munu þessar tvær aðferðir ávallt velja besta vendistak og klára að raða fylkinu á sem stystum tíma meðan aðferðin sem velur slembið skiptistak er óhentugri. \\

\noindent Þannig að með því að prófa\footnote{\texttt{SawtoothTime.java}} á þeim þremur Quick sort aðferðum sem eftir eru hraðann við að raða sagtannar fylki og röðuðu fylki fékkst að \texttt{s4} og \texttt{s7} tóku nokkurn vegin sama tíma að jafnaði í báðum tilfellum meðan \texttt{s10} tók mun skemmri tíma við að raða sagtannar fylkinu en mun lengri til að raða raðaða fylkinu. Í báðum tilfellum tóku hægari aðferðirnar um tvöfalt lengri tíma. Sjáum niðurstöðurnar í eftirfarandi töflu: \\

\begin{tabular}{c|c|c}

Aðferð & Raðað fylki [ns] & Sagtannarfylki [ns] \\ \hline
\texttt{s4} &   4,742,278   &   116,097,204\\
\texttt{s7} &   4,459,498   &   117,871,950\\
\texttt{s10} &  12,271,866  &   23,768,634\\
\end{tabular} \\

\noindent Við ályktum því að \texttt{s10} er Quick sort með sem velur skiptistak af handahófi. 

\subsubsection{Aðrar Quick sort aðferðir}
Vitað er að \texttt{s4} og \texttt{s7} eru þær Quick sort aðferðir sem annars vegar velja vendistakið sem miðjustakið og hins vegar miðgildið af fremsta, aftasta og miðjustaki. \\

\noindent Þegar aðferðirnar eru notaðar til þess að raða fylkinu \texttt{a = [1,2,3]} munu þær haga sér nákvæmlega eins: Velja miðjustakið sem vendistak, halda því á sínum stað og raða svo grunntilfellunum [1] og [3]. \\

\noindent Önnur aðferðin þarf að reikna \texttt{$\lfloor$3/2$\rfloor$ = 1}, vísa þrisvar í fylkið og finna miðgildi stakanna \texttt{a[0] = 1}, \texttt{a[1] = 2} og \texttt{a[2] = 3} til þess að velja vendistakið, á meðan hin þarf bara að reikna \texttt{$\lfloor$3/2$\rfloor$ = 1} og velja vendistakið sem \texttt{a[1] = 2}.\\

\noindent Þetta þýðir að aðferðin sem velur miðjustakið sem vendistak er fljótari að raða fylkinu \texttt{[1,2,3]} heldur en hin.\\

\noindent Þegar \texttt{s4} og \texttt{s7} voru látnar raða\footnote{\texttt{Comparer.c47()}} \texttt{[1,2,3]} endurtekið $2\cdot 10^9$ sinnum tók það $20291$ millisekúndur fyrir \texttt{s4} en $17135$ millisekúndur fyrir \texttt{s7}. Þessi tilraun gaf sömu niðurstöður í hvert sinn sem hún var endurtekin.\\

\noindent Við ályktum því að \texttt{s4} sé sú Quick sort aðferð sem velur sem vendistak miðgildi fremsta, aftasta og miðjustaks og að \texttt{s7} sé sú Quick sort aðferð sem velur miðjustakið sem vendistak.

%~ 1.3
\subsection{Aðferðir sem nota auka minni}
\subsubsection{Counting sort}
\noindent Counting sort tekur frá minni sem er jafn stórt og mismunur stærsta og minnsta staksins. Eina aðferðin sem kláraði minni sýndarvélarinnar við að raða fylkinu\footnote{\texttt{CountingProf.java}} \texttt{[1 , Integer.MAX\_VALUE]} var \texttt{s9}.\\

\noindent Því er ljóst að \texttt{s9} er Counting sort.

%~ 1.4
\subsubsection{Merge sort}
Ef Merge sort er notað til þess að raða \texttt{a = [M,M,M,M]}, þar sem \texttt{M = Integer.MAX\_VALUE} gerist eftirfarandi þegar kallað er á \texttt{MERGE(a,0,1,3)}:\\
\begin{itemize}
    \item   Fylkin \texttt{L = [M,M,M]} og \texttt{R = [M,M,M]} eru mynduð.
    \item   Breyturnar \texttt{i}, \texttt{j} og \texttt{k} upphafsstilltar sem 0 og \texttt{k} látið hlaupa upp í 3.
    \item   Segðin \texttt{L[i] <= R[j]} tekur alltaf gildið \texttt{true}, svo \texttt{i} er hækkað um einn í hverri umferð og er því jafnt \texttt{k}.
    \item   Í síðustu umferðinni er \texttt{i} orðið 3 og segðin \texttt{L[i] <= R[j]} verður ólögleg því \texttt{L} er þriggja staka fylki.
\end{itemize}
Þegar föllin \texttt{s5}, \texttt{s6} og \texttt{s8} eru keyrð\footnote{\texttt{Comparer.c568()}} með viðfangið \texttt{a} er \texttt{s8} það eina sem gefur\\ \texttt{java.lang.ArrayIndexOutOfBoundsException}. Þess vegna er \texttt{s8} Merge sort.\\

\subsubsection{Radix og Bucket sort}
Ef Bucket sort er notað til þess að raða fylki af lengd $n$ sem hefur töluna $3n$ í fremsta sæti og tölurnar 0 og 1 til skiptis í öðrum sætum gerist eftirfarandi:
\begin{itemize}
    \item Til verður fylki af $n$ fötum. Þar sem stærsta talan í inntaksfylkinu er $3n$ komast a.m.k. tvær mismunandi tölur í hverja fötu.
    \item Í fyrstu fötuna fara 0 og 1 til skiptis uns hún inniheldur $n-1$ tölur.
    \item Insertion sort er látið raða fyrstu fötunni, en það tekur $O(n^2)$ tíma.
\end{itemize}
Við fáum því keyrslutímann $O(n^2)$ fyrir Bucket sort fallið, en $O(n)$ fyrir Radix sort.\\

\noindent Vitað er að föllin \texttt{s5} og \texttt{s6} eru Radix sort og Bucket sort. Þau voru því látin raða fylki\footnote{\texttt{Comparer.c56()}} af umræddu tagi fyrir stækkandi $n$ og tímamælingar teiknaðar\footnote{\texttt{plot.r}} á eftirfarandi mynd. \\

\begin{figure}[h!]
    \centering
    \includegraphics[scale=0.65]{Radix_og_Bucket.png}
    \caption{Hallatala \texttt{s5} = 0.9647 \& hallatala \texttt{s2} = 1.871}
\end{figure}

\noindent Af myndinni má ráða að \texttt{s6} tekur $O(n^2)$ tíma og \texttt{s5} tekur $O(n)$ tíma.
Við getum því dregið þá ályktun að \texttt{s5} er Radix sort og \texttt{s6} er Bucket sort.

\section{Niðurstöður}
\begin{tabular}{ c | c }
    Sorter.class & Röðunarreiknirit \\ \hline
    \texttt{s1}  & Heap sort \\
    \texttt{s2}  & Insertion sort \\
    \texttt{s3}  & Quick sort: aftasta stak, n \\
    \texttt{s4}  & Quick sort: miðgildið \\
    \texttt{s5}  & Radix sort \\
    \texttt{s6}  & Bucket sort \\
    \texttt{s7}  & Quick sort: miðjustak, n/2 \\
    \texttt{s8}  & Merge sort \\
    \texttt{s9}  & Counting sort \\
    \texttt{s10} & Quick sort: slembistak \\ \hline
\end{tabular}

\end{document}
