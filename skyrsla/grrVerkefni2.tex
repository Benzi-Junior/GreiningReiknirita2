%~ xelatex -shell-escape grrVerkefni1.tex

\documentclass[a4paper,oneside]{article}
\usepackage{a4wide}
\usepackage[icelandic]{babel}
\usepackage{fontspec}
\usepackage{xunicode}
\usepackage{graphicx}
\usepackage{enumerate}

\usepackage{mathtools} % := from \coloneqq
\usepackage{units} % falleg 1/3 brot
\usepackage{amsmath} % \begin{align} (also in \{mathtools})
\usepackage{amsthm} % proof
\usepackage{amssymb} % blacksquare og mathbb
 \renewcommand{\qedsymbol}{$\blacksquare$}

%~ Teikna tré
\usepackage{qtree}

\usepackage{fancyhdr}
\pagestyle{fancy}

\lhead{Greining reiknirita}
\chead{Verkefni 1}
\rhead{Háskóli Íslands}

\title{
    Greining reiknirita:\ Verkefni 1
    \\\small{Kennari: Páll Melsted}
}
\author{Bjarki Geir Benediktsson, \and  Bjarni Jens Kristinsson \and og Tandri Gauksson}
\date{\small{Skil:\ 9. febrúar 2014}}


\begin{document}
\maketitle

\section{Verklýsingin}
Verkefnið var að hanna gagnagrind sem notaðist við einhverskonar tré til að geyma lokuð bil  á rauntalnalínunni þar sem báðir endanpunktarnir eru heiltölur.
Við notuðum AVL tré til að geyma bilin sérhver hnútur inniheldur bil ásamt hæsta og lægsta endapunkt sem fyrirfynnst í undirtrénu með þann hnút sem rót.
 


\end{document}
